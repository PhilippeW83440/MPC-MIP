%% LyX 2.3.3 created this file.  For more info, see http://www.lyx.org/.
%% Do not edit unless you really know what you are doing.
\documentclass{article}
\usepackage[T1]{fontenc}
\usepackage[utf8]{inputenc}
\usepackage{amsmath}
\usepackage{graphicx}
\usepackage{microtype}
\usepackage[unicode=true,
 bookmarks=false,
 breaklinks=false,pdfborder={0 0 1},backref=section,colorlinks=false]
 {hyperref}

\makeatletter
%%%%%%%%%%%%%%%%%%%%%%%%%%%%%% User specified LaTeX commands.
\usepackage[final]{nips_2017}
% allow utf-8 input
% use 8-bit T1 fonts
% hyperlinks
\usepackage{url}% simple URL typesetting
\usepackage{booktabs}% professional-quality tables
\usepackage{amsfonts}% blackboard math symbols
\usepackage{nicefrac}% compact symbols for 1/2, etc.
% microtypography
\title{Trajectory Optimization with Dynamic Obstacles Avoidance}

\author{
  Philippe Weingertner and Minnie Ho\\
  \texttt{pweinger@stanford.edu minnieho@stanford.edu} \\
  %% examples of more authors
  %% \And
  %% Coauthor \\
  %% Affiliation \\
  %% Address \\
  %% \texttt{email} \\
  %% \AND
  %% Coauthor \\
  %% Affiliation \\
  %% Address \\
  %% \texttt{email} \\
  %% \And
  %% Coauthor \\
  %% Affiliation \\
  %% Address \\
  %% \texttt{email} \\
  %% \And
  %% Coauthor \\
  %% Affiliation \\
  %% Address \\
  %% \texttt{email} \\
}

\usepackage{algorithm,algpseudocode}


\usepackage{tikz}
\usetikzlibrary{shapes, arrows}
\usetikzlibrary{er,positioning}
\usetikzlibrary{matrix}
\tikzset{
    events/.style={ellipse, draw, align=center},
}

\usepackage{graphicx}
\usetikzlibrary{fit}
\usetikzlibrary{bayesnet}
\usepackage{pgfplots}

\usepackage{forest}

\makeatother

\begin{document}
% \nipsfinalcopy is no longer used

\maketitle
\begin{abstract}
We study the problem of Trajectory Optimization. 
\end{abstract}

\section{Introduction}

This project  investigates trajectory optimization in the presence
of obstacles \cite{article,Fan2018BaiduAE,Katrakazas2015RealtimeMP,Zhang2020OptimizationBasedCA}.
One such application for this class of problems is that of autonomous
driving, where we have an ego vehicle and dynamic obstacles (vehicles,
pedestrians) which may intersect our desired trajectory and which
we wish to avoid using motion planning and control. Trajectory optimization
problems minimize a cost function which takes into account start and
terminal states, as well as cost along the trajectory path. The design
space is subject to constraints on the states and control input at
sampled time points.

\section{Related Work}

Todo ...

\section{Problem Formulation}

We define a MPC problem over 20 time steps of $250$ ms each\textbf{
}with a Quadratic Cost function with $x\in\mathbb{R}^{60}$ and $160$
constraints. We have $120$ linear and nonlinear ($\left\Vert x_{\text{ego}}-x_{\text{obj}}\right\Vert \geq d_{\text{saf}}$)
inequality constraints and  $40$ linear equality constraints (Dynamics
Model).

\[
\underset{u_{0},\ldots,u_{T-1}}{\min}\left(x_{T}-x_{\text{ref}}\right)^{\intercal}Q_{T}\left(x_{T}-x_{\text{ref}}\right)+\sum_{k=0}^{T-1}\left(x_{k}-x_{\text{ref}}\right)^{\intercal}Q\left(x_{k}-x_{\text{ref}}\right)+u_{k}^{\intercal}\;R\;u_{k}
\]

$\text{subject to }\begin{cases}
x_{k,\text{min}}\leq x_{k}\leq x_{k,\max}\\
u_{k,\text{min}}\leq u_{k}\leq u_{k,\max}\\
x_{k+1}=A_{d}x_{k}+B_{d}u_{k}\\
x_{0}=x_{\text{init}}\\
\forall\left(t_{\text{col}},s_{\text{col}}\right)_{i\in\left[1,10\right]} & x_{t_{\text{col}}^{\left(i\right)}}\left[1\right]<s_{\text{col}}^{\left(i\right)}-\Delta_{\text{safety}}\text{ or }x_{t_{\text{col}}^{\left(i\right)}}\left[1\right]>s_{\text{col}}^{\left(i\right)}+\Delta_{\text{safety}}
\end{cases}$

\bigbreak	Linear Dynamics with Constant Acceleration model in between
2 time steps

\[
\begin{bmatrix}s\\
\dot{s}
\end{bmatrix}_{k+1}=A_{d}\begin{bmatrix}s\\
\dot{s}
\end{bmatrix}_{k}+B_{d}\begin{bmatrix}\ddot{s}\end{bmatrix}_{k}\text{with }A_{d}=\begin{bmatrix}1 & \Delta t\\
0 & 1
\end{bmatrix},B_{d}=\begin{bmatrix}\frac{\Delta t^{2}}{2}\\
\Delta t
\end{bmatrix}
\]


\section{Methods }

\subsection{Collision Avoidance Model}
\begin{itemize}
\item Elastic Model
\item Disjunctive Constraints Handling
\end{itemize}

\subsection{Optimization Algorithms}

\subsection{Penalty Methods}

\[
p_{\text{quadratic}}\left(x\right)=\sum_{i}\max\left(g_{i}\left(x\right),0\right)^{2}+\sum_{j}h_{j}\left(x\right)
\]

\[
p_{\text{Lagrange}}\left(x\right)=\frac{1}{2}\rho\sum_{i}h_{i}\left(x\right)^{2}-\sum_{i}\lambda_{i}h_{i}\left(x\right)
\]


\subsubsection{Interior Point Method with Inequality and Equality Constraints}

\[
\underset{\text{subject to}}{\min}\begin{cases}
\hat{f}\left(x+v\right)=f\left(x\right)+\nabla f\left(x\right)^{T}v+\frac{1}{2}v^{T}\nabla^{2}f\left(x\right)v & \text{Taylor 2nd order approx}\\
A\left(x+v\right)=b
\end{cases}
\]
 
\[
\text{Via optimality conditions on \ensuremath{\mathcal{L}\left(x,\lambda\right):} }\begin{bmatrix}\Delta x_{\text{newton\_step}}\\
\lambda
\end{bmatrix}=\begin{bmatrix}\nabla^{2}f\left(x\right) & A^{T}\\
A & 0
\end{bmatrix}^{-1}\begin{bmatrix}-\nabla f\left(x\right)\\
-\left(Ax-b\right)
\end{bmatrix}
\]


\subsubsection{Simplex Algorithm}

Simplex is fast. We investigate how to bootstrap the feasibility search
phase of an Interior Point method with a simplex algorithm.

\subsection{Optimization under Uncertainty}

\[
\underset{x\in\mathcal{X}}{\min}\quad\underset{z\in\mathcal{Z}}{\max}\quad f\left(x,z\right)
\]


\section{Experiments}

The gihub repo is \href{https://github.com/PhilippeW83440/AA222_Project}{AA222-project}. 

\subsection{ST Graphs Analysis}
\begin{itemize}
\item Runtime: $\leq250$ ms for real time applicability
\item Feasibility constraints compliance: check safety \& dynamics constraints
\item Cost value: efficiency and comfort (lower cost function)
\end{itemize}
\begin{center}
\includegraphics[scale=0.3]{/home/philippew/AA222_Project/writeup/presentation/path1d_interior1_st_test1}
\par\end{center}

\subsection{Anti Collision Tests Benchmarks}

We use five metrics to evaluate the performance of our different approaches.
$\left(1\right)$ The main success metric is the percentage of cases
where we reach a target state without collision. $\left(2\right)$
The second metric is the agent runtime. $\left(3\right)$ The third
metric is a comfort metric: the number of hard braking decisions.
$\left(4\right)$ The fourth metric relates to efficiency: how fast
we reach a target while complying to some speed limitation. $\left(5\right)$
The last metric is a safety metric: for some of our randomly generated
test cases, a collision is unavoidable. In these cases, we aim for
a lower speed at collision.

\section{Conclusion }

It works even better than expected ...

\bibliographystyle{plain}
\nocite{*}
\bibliography{project}

\end{document}
